% interactapasample.tex
% v1.05 - August 2017

\documentclass[]{interact}

\usepackage{epstopdf}% To incorporate .eps illustrations using PDFLaTeX, etc.
\usepackage[caption=false]{subfig}% Support for small, `sub' figures and tables
\usepackage{url}
\usepackage[hyphenbreaks]{breakurl}
%\usepackage[nolists,tablesfirst]{endfloat}% To `separate' figures and tables from text if required
%\usepackage[doublespacing]{setspace}% To produce a `double spaced' document if required
%\setlength\parindent{24pt}% To increase paragraph indentation when line spacing is doubled

% \usepackage[longnamesfirst,sort]{natbib}% Citation support using natbib.sty
% \bibpunct[, ]{(}{)}{;}{a}{,}{,}% Citation support using natbib.sty
% \renewcommand\bibfont{\fontsize{10}{12}\selectfont}% To set the list of references in 10 point font using natbib.sty

\usepackage[natbibapa,nodoi]{apacite}% Citation support using apacite.sty. Commands using natbib.sty MUST be deactivated first!
\setlength\bibhang{12pt}% To set the indentation in the list of references using apacite.sty. Commands using natbib.sty MUST be deactivated first!
\renewcommand\bibliographytypesize{\fontsize{10}{12}\selectfont}% To set the list of references in 10 point font using apacite.sty. Commands using natbib.sty MUST be deactivated first!
\def\UrlBreaks{\do\/\do-}

\theoremstyle{plain}% Theorem-like structures provided by amsthm.sty
\newtheorem{theorem}{Theorem}[section]
\newtheorem{lemma}[theorem]{Lemma}
\newtheorem{corollary}[theorem]{Corollary}
\newtheorem{proposition}[theorem]{Proposition}

\theoremstyle{definition}
\newtheorem{definition}[theorem]{Definition}
\newtheorem{example}[theorem]{Example}

\theoremstyle{remark}
\newtheorem{remark}{Remark}
\newtheorem{notation}{Notation}

\begin{document}

\articletype{REGISTERED REPORT RESEARCH REPLICATION STUDY}% Specify the article type or omit as appropriate

\title{An Empirical Replication of ``Computing in the curriculum:
  Challenges and strategies from a teacher’s perspective''}

\author{
\name{Tom Crick\textsuperscript{a}\thanks{CONTACT Professor Tom
    Crick. Email: thomas.crick@swansea.ac.uk} and Cathryn Knight\textsuperscript{a}}
\affil{\textsuperscript{a}Swansea University, UK}
}

\maketitle

% % a structured abstract (250 word max).  The journal requires a
% Structured abstract for papers; for the stage 1 submission you will
% supply background, objective and method, but leave the findings and
% implications blank.
\begin{abstract}
{\emph{Background:}} There have been rapid changes to Computing
education across all four nations of the UK over the past ten years.
In 2014,~\citeauthor*{sentance+csizmadia:2017} investigated teachers’
perceptions of the compulsory inclusion of Computing into the new
national curriculum in England using a large-scale survey. The results
of this national case study have been widely-cited in the literature,
categorising approaches taken by Computing teachers to support
students.  However, while~\citeauthor*{sentance+csizmadia:2017}'s
\citeyear{sentance+csizmadia:2017} paper provides valuable insight
into teachers’ perceptions from 2014, the data collected was largely
from England. Therefore, it did not account for the devolved
educational policy contexts across the UK, and the policy and practice
changes in the interim.
 
{\emph{Objective:}} The proposed project replicates the original study
to find out both how perceptions have changed between 2014 and the
present day, and to see how they may be further impacted by diverging
UK Computing education policy and practice contexts. Furthermore, the
replication will also include specific questions relating to the
COVID-19 pandemic in order to better understand its impact on
Computing education, and to control for any changes to learning,
teaching and assessment that have taken place due to the pandemic.
 
{\emph{Method:}} We propose an empirical replication
of~\citeauthor*{sentance+csizmadia:2017}'s
\citeyear{sentance+csizmadia:2017} original study, using the same
online survey. Both quantitative and qualitative data will be
collected in order to better understand the changes in teachers’
perceptions toward Computing education, and identify any differences
across the four nations of the UK.

\end{abstract}

\begin{keywords}
Computer science education; in-service teacher education; computing
curricula; registered report; replication study
\end{keywords}


% an introduction, in which the authors clearly justify the replication
% of the original study and the specific aspects they will add or modify
% to shed more light on the phenomenon studied. This section should also
% include hypotheses about how the outcomes of the replication will
% compare to the original study.
\section{Introduction}

In this Stage 1 Registered Report, we outline our proposal for the
empirical replication of~\citeauthor*{sentance+csizmadia:2017}'s
highly-cited\footnote{Citation metrics taken from Google Scholar:
\url{https://scholar.google.com/scholar?cluster=5696382193951950255}
(155 citations as of November 2020)} paper ``Computing in the
curriculum: Challenges and strategies from a teacher’s
perspective''\footnote{See:
\url{https://doi.org/10.1007/s10639-016-9482-0}}, published in the
Springer journal {\emph{Education and Information Technologies}
in~\citeyear{sentance+csizmadia:2017}.

Computing is being introduced into the curricula in many nations,
regions and jurisdictions. Teachers’ perspectives of these major
reforms allow us to discover and better understand what challenges --
and opportunities -- this presents, and also the strategies teachers
claim to be using in teaching the subject across primary
and secondary (K-12) education. The study described in this paper was
originally carried out in the United Kingdom (UK) in 2014, in the
midst of major Information and Communication Technology (ICT) national
curriculum reform in England, where teachers were preparing for the
mandatory inclusion of Computing into the curriculum. We have since
seen major national curriculum reforms published in Wales in January
2020, alongside increased scrutiny of existing curricula in Scotland
and Northern Ireland, as well as the quality and perceived value of
technical school-leaver qualifications across all four nations.

In 2014,~\citeauthor*{sentance+csizmadia:2017} launched a survey
investigating teachers’ perceptions of the compulsory inclusion of
Computing into the new national curriculum in England. N=1417
respondents completed the wider survey (1126 of whom were practising
teachers), with n=339 teachers contributing at least one free text
answer to the free text questions. In the paper,
~\citeauthor*{sentance+csizmadia:2017} primarily focused on the n=339
responses given by this self-selecting group of teachers, but include
reference to their other answers to survey questions where
relevant. Using Likert-scale questions, they found that the majority
of teachers (85\%) rated their confidence in being able to deliver the
new curriculum at 6 or more out of 10~\citep{sentance+csizmadia:2017}.
However, this statistic was reported from a subsample of the
participants who answered open-ended questions about teaching
Computing at school.

Therefore, it is necessary to return to the 2014 data in order to
determine the confidence levels within the wider sample of 1,126
practicing teachers contacted through the Computing at School
(CAS)\footnote{\url{https://www.computingatschool.org.uk/}} UK
membership association. Within the qualitative data teachers reported
a number of intrinsic and extrinsic challenges in teaching
Computing. Statements made by teachers who are currently teaching
Computing in school were coded, categorised and analysed, describing
both successful strategies for teaching and the difficulties they
face. These included the need to change teaching strategy in order to
account for the differences between ICT and Computing; concerns over
how to teach computational thinking; and, the need to develop
resilience in students when learning programming. Overall, the paper
offers guidance to teachers on how develop their Computing teaching
skills. The results reported in the paper were timely and relevant to
teachers and teacher educators in the field of Computing, especially
during the midst of major national curriculum reform in England. Their
recommendations include the need for further investigation to be
carried out around the impact of the strategies suggested in primary
and secondary education~\citep{sentance+csizmadia:2017}. Therefore,
the replication of this study would also reinforce contributions to
the area of pedagogical content knowledge (PCK) in Computing; PCK is
the knowledge that a teacher has about how to teach their
subject~\citep{shulman:1986}.

Yet, while ~\citeauthor*{sentance+csizmadia:2017}'s
\citeyear{sentance+csizmadia:2017} paper provided a valuable insight
into UK teachers’ perceptions from 2014 (which has garnered significant
international attention e.g.~\citet{nytcoding:2014}), the data
collected was largely from England. Therefore, it largely did not
account for the devolved educational policy context across the UK, and
the policy and practice changes in the interim. Furthermore, while
N=1,126 practicing teachers completed the survey, only n=339 completed the
qualitative questions on which the findings were
based. \citeauthor*{sentance+csizmadia:2017} note that the sample they
draw upon ``may not be `typical' of the whole teacher population, but
represent teachers who are more comfortable teaching Computing''
(p. 477). This makes it difficult to generalise the results to the
broader population of UK teachers. Therefore, a major justification
for an empirical replication of this research study is to explore how
opinions have changed between 2014 and now, and also to gather a more
representative sample of practitioners from across the four nations of
the UK in order to better understand the impacts of the differing and
evolving policy and professional practice contexts.

Through this empirical replication study, we anticipate a more
granular response to the original survey instrument, reflecting the
increased differences in Computing education policy and practice, as
well as the divergence in national curricula and qualifications across
the UK. However, we would anticipate a larger range of positive and
negative responses, reflecting the challenges of recruiting, retaining
and professionally developing Computing teachers across the UK,
potentially developing distinct teacher grouping of ``haves'' and
``have nots'' in their confidence and capability to deliver specific
curriculum and qualifications. By aiming for a UK-representative
survey sample, it would provide an opportunity for comparative
thematic analysis between the four nations of the UK, to provide
deeper insight that may be portable to other nations and
jurisdictions.

Finally, the impact of the COVID-19 pandemic on the wider education
system, across all settings, has been
profound~\citep{unescocovidedu:2020}, presenting significant
challenges for learning, teaching and assessment
(LT\&A)~\citep{oecd:2020}. Across the UK, there have been major
responses from various governments, organisations and institutions at
all levels and settings; from major national policy initiatives to
support learners and maintain quality and standards across all
settings, to ongoing government inquiries on the longer-term impact of
COVID-19 on education and children’s services. Thus, we are keen to
better understand the short and medium-term impact of COVID-19 on
Computing educators at all levels and settings in the UK, given some
of the specific disciplinary challenges of teaching Computing,
building on recent work in this
space~\citep{crick-et-al:ukicer2020,watermeyer-et-al:he2020}, as well
as the opportunities for wider education system
change~\citep{zhao:2020}; for example, the future of formal
examinations and qualifications in the UK.

\subsubsection*{A Note on Terminology}

While in many instances throughout this paper we will refer to the UK
-- consisting of the four nations of England, Scotland, Wales and
Northern Ireland -- we will attempt to be as clear as possible when
referring to specific policies or initiatives across or between the
nations, as a number of policy areas, including education and skills,
are devolved to the respective national governments.

With regards to the consistent naming of the discipline through this
paper, ``Computing'' is used as the subject name throughout this paper
as well as in~\citet{sentance+csizmadia:2017}; it refers mainly to the
computer science elements of the curriculum, and is also often
referred to as Informatics in other countries.

% a literature review, containing a summary of the relevant literature,
% especially the literature which has been published since the
% publication of the original paper.
\section{Brief Literature Review}

In addition to the review of the key literature
in~\citet{sentance+csizmadia:2017}, including the theoretical framing,
much has developed in a policy and practice context for Computing
education across the UK.

Computing has been introduced as a new subject in compulsory-level
school curricula in many countries, with broad social, cultural and
economic objectives~\citep{tuomi-et-al:2018}. This brings with it both
excitement and challenges, as for any new subject; for teachers facing
curriculum change, how to confidently teach it is very
pertinent. Introducing new content does not merely mean that teachers
have to equip themselves with new subject knowledge, which of course
in many cases they
do~\citep{brown-et-al-sigcse2013,sentance-et-al:2013,thompson+bell:2013,sentance+humphreys:2017}.
Teachers also need to learn appropriate pedagogies for delivering a
new subject, particularly in those aspects of computer science that
relate to algorithms, programming and the development of computational
thinking
skills~\citep{davenport-et-al:latice2016,murphy-et-al:programming2017,kong-et-el:2020}.

The UK has seen rapid change -- in both curricula and qualifications
-- in the area of computer science education in recent years, across
all four nations of the
UK~\citep{brown-et-al-sigcse2013,brown-et-al-toce2014,cutts-et-al:2017,moller+crick:jce2018},
with education being a devolved responsibility for the individual
governments, parliaments and assemblies. Before the major national
curriculum reforms from September 2014, many schools and teachers in
England had implemented elements of the new Computing curriculum prior
to the official starting date of the Computing Programme of
Study~\citep{dfecomputingpos:2013} in September 2014, as a void was
left by the disapplication of the existing curriculum subject, ICT, in
January 2012~\citep{brown-et-al-toce2014}. We have also seen major
curriculum reforms in Wales, published in January 2020, with
substantial shifts from ICT to Computing, the recognition of statutory
cross-curricular digital competencies, and the development of a new
Science \& Technology area of learning and
experience~\citep{cfw:2020}. Alongside these curricula reforms, we
have also seen significant changes to the available technical
school-leaver qualifications, taken at the ages of 16 and 18.

Recent literature relating to Computing education in school
highlights a number of ways of making computer science concepts
accessible, engaging and fun, and more importantly, giving students a
deep understanding of these concepts. This should be grounded in
appropriate educational theory and methods, directly supporting
effective professional practice. Thus, the implementation of a new
Computing curriculum involves a change to teachers’
practice~\citep{thompson+bell:2013,sentance+humphreys:2017,
moller+crick:jce2018} across both their subject knowledge and
pedagogical knowledge. With respect to the introduction of technology
in classrooms, a different context, it can be seen that there is an
intersection between teachers’ knowledge, beliefs and
culture~\citep{ertner+ottenbreit-leftwich:2010,voogt+mckenney:2017}
and this may indeed be the same for Computing. This further draws on
the work of~\citet{finger+houguet:2009} who, also working in the area
of adoption of technology into the curriculum, describe a range of
intrinsic and extrinsic challenges that teachers face in moving from
the intended to the implemented
curriculum~\citep{vandenakker:2004,vinnervik:2020}.

Prior to the work of ~\citet{sentance+csizmadia:2017},
\citet{black-et-al:2013} carried out a study in the UK where they
asked Computing teachers how they felt they could make the subject
interesting. The key aspects that they identified were the importance
to teachers of making Computing fun and relevant. In carrying out this
replication study, we are interested to see whether the teachers’
comments aligned with both the \citeauthor*{black-et-al:2013} and
\citeauthor*{sentance+csizmadia:2017} studies; we thus wish to
further validate actual strategies that teachers use in their
classroom that they feel to be most effective for teaching Computing.

In a similar way to the \citet{black-et-al:2013} study, this
replication study of \citet{sentance+csizmadia:2017} will focus purely
on the teacher’s perspective in addressing these important
questions. \citet{diethelm-et-al:2012} emphasise the importance of the
teacher’s perspective to our understanding of computer science
education as the teacher ``may work on many different abstraction
levels or apply very different teaching methods for the same topic of
the curriculum'' (p. 167), which is reaffirmed by later
work~\citep{bender-et-al:2015,hubbard:2018}. We wish to identify what
these methods are, in particular identifying common themes that may
help to provide guidance for teachers new to teaching the subject, as
well as providing actual examples of teachers using effective
strategies as we enter a phase of education where more and more
students are studying Computing in school, across all four nations of
the UK, with significant opportunities to shape evolving Computing
education policy and practice.

% a method section, defining the type of replication that will be used
% (see below), clearly distinguishing between the materials/approaches
% of the original study and the additions of the replication study.
\section{Methods}

We propose an empirical replication of the original study
from~\citet{sentance+csizmadia:2017}. We have secured access from the
authors to the original survey, and both the quantitative and
qualitative data to be able to provide comparative analysis to the
2014 results.

Further to the two research questions outlined in ~\citet{sentance+csizmadia:2017}:

\begin{itemize}
\item {\emph{What pedagogical strategies do teachers report work well for teaching
computer science in school?}}

\item {\emph{What challenges do teachers report that they face?}}
\end{itemize}

We propose the following three additional research questions as part of this
empirical replication study:

\begin{itemize}
\item How have teachers' perceptions of Computing in the curriculum changed
between 2014 and 2021?

\item How do teachers’ perceptions of Computing in the curriculum differ by
UK country?

\item Has COVID-19 had an impact on teachers’ perceptions and confidence
teaching Computing?
\end{itemize}

\subsection{Sample}

Replicating the original study, this research will use an online
survey targeting the UK membership of the Computing at School (CAS) UK
network, as well as convenience sampling through social media
platforms such as Twitter and LinkedIn. Furthermore, in order to
ensure a representative sample from across the UK, CAS-related
networks to CAS will be used in Wales, Scotland and Northern Ireland
through professional contacts and policy links.

\subsection{Survey}

The data will be collected using the Qualtrics online survey
software. The survey will follow the same structure as was used
in~\citet{sentance+csizmadia:2017}, therefore it will collect
demographic questions, followed by the same Likert and open-ended
questions. Furthermore, in order to address the research question on
whether there has been an impact of COVID-19 on teaching Computing,
the survey will contain questions specifically relating to whether the
rapid transition to online learning, teaching and assessment has
impacted teacher confidence in teaching Computing.

\subsection{Original survey}

The data from the original~\citet{sentance+csizmadia:2017} survey will
be used in order to determine whether there are significant
differences between the original 2014 data, and the new survey data. Both
quantitative and qualitative data has been provided by the original
authors in order to undertake this analysis.

% a plan of analyses, outlining the original analyses and the newly
% planned analyses. Quantitative replications should also include a
% power analysis in this section.
\section{Plan of Analyses}

\subsection{Quantitative analysis}

In order to understand whether there is a significant difference
between the original data from 2014 and the current survey, bivariate
time 1 to time 2 analysis will be conducted. The data for this
analysis will be weighted post-hoc to ensure that the demographics
mirror the demographics of the original survey. T-tests will be used
to determine differences in how the amount of time teaching Computing
has changed since~\citeauthor*{sentance+csizmadia:2017} originally collected
the data. Chi-square tests ($\chi^2$) will be used to see if there is
a significant difference between the confidence levels of teachers
between time 1 and time 2.

Secondly, to determine differences between countries of the UK,
$\chi^2$ tests will be used to compare confidence levels. $\chi^2$
tests will determine whether educators in each country are
significantly more or less likely to state that they are confident
teaching Computing.

Finally, to understand the impact of COVID-19 on teacher confidence,
univariate and bivariate $\chi^2$ analysis will be used to present the
results of the COVID-19 specific questions and to understand whether
there was significant difference in responses based on respondent
demographic and country.

For all quantitative analysis significance will be indicated by a p
value $<0.05$.

\subsection{Qualitative Analysis}

In the original study, the following open-ended survey questions were
used to gather qualitative data from practitioners on Computing in the curriculum:  

\begin{enumerate}
\item {\emph{What good techniques/strategies have you found for helping
students to understand programming?}}

\item {\emph{Please describe any good techniques/strategies you use for
helping students to understand other aspects of Computing?}}

\item {\emph{What difficulties, if any, have you experienced teaching
programming?}}

\item {\emph{What difficulties, if any, have you experienced teaching other
aspects of Computing?}}
\end{enumerate}

Using a similar method to~\citeauthor*{sentance+csizmadia:2017}, the
qualitative data will be coded and placed into themes. However, rather
than using a bottom-up method of coding, original themes from
~\citet{sentance+csizmadia:2017} will be identified in the data. Where
an entry does not fit the original code, a new code will be
generated. This will allow for comparisons between the 2014 and 2021
data. The data will be coded twice and inter-rater reliability using
the Kappa statistic will be performed.

% risks to the study, including: whether the funding/means to conduct
% the study is already secured or is pending, the current status of the
% ethics application, and any likely barriers to completing the study
% before the end of 2021 (see timeline below).
\section{Risks}

This project has already received ethics approval through Swansea
University's Faculty of Social Sciences and Humanities ethics board
(approval reference: {\emph{SU-Ethics-Staff-151120/295}}). No funding
is being sought for this research, and there is no dependence on
employing additional researchers to complete this proposed
replication.

One identified risk to the proposed replication study is related to
the current environment in which the data will be collected, due to
the ongoing COVID-19 context. It is possible that compulsory
(school-level) education may again return to online learning, teaching
and assessment due to the continued impact of COVID-19, and that the
approach may differ across the four nations of the
UK~\citep{crick-et-al:ukicer2020,watermeyer-et-al:he2020}. These
different educational environment contexts may impact the responses to
the questions; however, it is anticipated that by asking specific
additional questions about the impact of COVID-19 that this effect can
be identified in the in the analysis and be controlled for.

Finally, we intend to obtain a UK-representative sample for this
replication study, but this may not be possible due to the chosen
sampling methods and prevailing COVID-related factors. Furthermore, we
will again recruit teachers through the CAS UK network, who already
have access to a lively and supportive grass-roots community of
teachers with whom they can exchange ideas and classroom resources. To
mitigate this, we will also attempt to recruit participants outside of
the (predominantly English) CAS membership, using similar professional
networks in Scotland, Wales and Northern Ireland, as well as using
convenience sample via social media (for example, Twitter and
LinkedIn).

% BibTeX
\bibliographystyle{apacite}
\bibliography{cse_rrrs_2021}

\end{document}
